\documentclass[11pt]{article}
\usepackage[top=1.00in, bottom=1.0in, left=1in, right=1in]{geometry}
\renewcommand{\baselinestretch}{1.1}
\usepackage{graphicx}
\usepackage{natbib}
\usepackage{amsmath}
\usepackage{parskip}

\def\labelitemi{--}
\parindent=0pt

\begin{document}
\bibliographystyle{/Users/Lizzie/Documents/EndnoteRelated/Bibtex/styles/besjournals}
\renewcommand{\refname}{\CHead{}}

\setlength{\parindent}{0cm}
\setlength{\parskip}{5pt}

8 May 2024: Notes from meeting at KWEER 

phylogenetic JC -- which exists
pull masting to community perspective ... needs phylogeny and pulls towards pest/pathogen host specificity 

JC and masting are both ways to escape pathogens (via space, JC ... and via time masting). But one is focused on negative DD and one is focused on positive DD (but the masting is probably cycling over time between positive and negative). 
- So JC should consider temporal aspect more
- Both should care more about host specificity of pathogens (with phylogeny of hosts)

Masting is a strategy to escape pathogens, JC you're stuck with pathogens and escape in space -- the strategy then should be to evolve dispersal (back to animal dispersal)... 

Masting people -- are REALLY interested in why trees mast (and how it matters to consumers) and maybe this could all connect to rarity/commonness and coexistence. 

JC people should think if masting explains anything. 

Janzen-Connell (conspecific negative density dependence CNDD) -- spatial density dependence that leads to coexistence. 

Masting (an economy of scale) -- predicts positive density dependence over time  
- If you are doing it asynchronously then you need host specificty.
- If you do it synchronously you can have generalist consumers.
- In between you can maybe less synchronous if hosts are phylogenetically distant. See Parker et al. 2015: `Phylogenetic structure and host abundance drive disease pressure in communities' which cites Janzen-Connell. 

... both verbal models. They only focus on trees, because you need a longer generation time (seed to reproductive maturity) to be longer than the predators can cycle. (They JC people think it works beyond trees -- e.g., grasslands -- and work on trees only because it's cool to explain all the trees.)

Masting as a way to escape JC? JC means whatever is common gets attacked, so if you mast you can avoid this and stay common. Most tropical forest communities are dominated by a couple species, so we predict they mast? [QUESTION to add.]

JC says many species in tropics because so many specialized consumers (pathogens) ... and CNDD is stronger in tropics, but now we say little CNDD in temperate zone. (? what did O mean)
Hulsman paper % https://www.nature.com/articles/s41586-024-07118-4
Zhang phylo paper % https://besjournals.onlinelibrary.wiley.com/doi/full/10.1111/1365-2745.13879

JC has recently integrated plant soil feedbacks better. 
See also `Feedback with soil biota contributes to plant rarity and invasiveness in communities' by Klironomos. 

Phylogenetic JC has clear signal. 

Latitudinal gradients... 
	- JC helps explain latitudinal gradient, but evidence is pretty weak; if masting is more prevalent in northern latitudes, then you might resolve this lack of evidence 
	- if masting species are more prevalent outside the tropics then they escape CNDD and this explains latitudinal gradient in diversity (but -- you may think, `hey there's no evidence of less CNDD in temperate' and we say -- people have not looked at space and time properly)
	
If there is phylogenetic signal in pest/pathogen host range then it would select for masting synchrony in those species; so phylogenetic signal currently observed in masting would be RECENTLY evolved (convergent evolution, not common ancestor) ... assumes species overlap in geographic ranges and co-occurrence in a community. 
	
- Seasonality slows down build of soil pathogens after masting? Is it easier to burn out pathogens in the tropics when the pathogens may be less likely to have dormancy stages. So could seasonal (within-year) seed production in the tropics be masting (burns out pathogens) effectively? And thus people don't fully recognize masting in tropics? MERGE this with, is it easier to evolve masting if you're seasonal already.

- Evolutionary inertia of seasonal cues makes it easier to evolve masting cues? 

% Leaving this out probably (conversation got nowhere): One EOS for masting is pollination efficiency (and does not intersect with JC). In the tropics everything is animal pollinated maybe because (1) rarity, (2) trees have leaves all the time so no wind work and (3). See Annual Review: % https://www.annualreviews.org/content/journals/10.1146/annurev.es.13.110182.002433

%JC: Is density or distance? 

Spatial and temporal scale of pathogens/consumers. (Add to host specificity... people need to include this in empirical tests.) When masting works depends on this AND the host specificity. Predict spatial scale of synchrony and depending on the generation time/mobility of consumers. Do JC and/or masting make implicit assumptions of mobility and generation time of consumers that each make? 

FIGURES:
- Two separate figures: one for JC people which is seeds per m2 versus survival (spatial) and for the masting it's seeds per m2 versus survival and temporal 
- Three dimensional to combine masting and JC: one axis is distance to mother tree, one is time and one survival
- Three dimensional: pathogen pressure versus time (time since masting) and space (distance from mother tree) -- see PP3D figure below

Hey, you guys explaining masting, could help explain community diversity.
And you guys explaining community diversity could explain masting. 

What do we want people to do from this paper? 
- Stop measuring saplings only; look at germinants. 
- Include phylogeny when thinking of host specificity. 
- So JC should consider temporal aspect more (from above) 
- Both should care more about host specificity of pathogens (with phylogeny of hosts) (from above)
- Develop modeling tools that have space and time. 

% Storage effect: good and bad environments... 
% Check Usinowicz masting paper...

PP3D figure
JC studies have no time, only space, and EOS Masting have looked over time. So both fields of study have ignored one important axes (either of space or time), which means it is difficult to understand the entire landscape

Parallel literatures! 


Paper structure:
- Outline the dichotomy between these fields -- similar processes, different dimenions, different outcomes assumes
- What would happen if we brought them together? Insert PP3D and predictions.
- Things that might get resolved
- Outstanding question. 


% Crazier stuff...
- It's easier to develop cues for masting in seasonal habitats?
- 

\end{document}

People who work on this ...
- Ingrid Parker
- Ian Pearse
- Liza Comita

