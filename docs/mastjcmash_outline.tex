\documentclass[11pt]{article}
\usepackage[top=1.00in, bottom=1.0in, left=1in, right=1in]{geometry}
\renewcommand{\baselinestretch}{1.1}
\usepackage{graphicx}
\usepackage{natbib}
\usepackage{amsmath}
\usepackage{parskip}
\usepackage{hyperref}


\def\labelitemi{--}
\parindent=0pt

\begin{document}
\bibliographystyle{/Users/Lizzie/Documents/EndnoteRelated/Bibtex/styles/besjournals}
\renewcommand{\refname}{\CHead{}}

\setlength{\parindent}{0cm}
\setlength{\parskip}{5pt}

\title{The missing mash-up that explains tree biodiversity, masting and why there are so many squirrels some years and not others: \\ The Janzen Masting Connell Hypothesis}
\author{Janneke, Jonathan, Lizzie}
\date{\today}
\maketitle
\tableofcontents

\section{Main Text draft}

Pest and pathogen pressure has been linked to negative density dependence, favouring higher tree diversity in tropical forests through J-C effects, and selection for masting – the synchronous and episodic investment in reproduction among years. Despite these two phenomena sharing a common driver and origins in the literature, they have been largely explored independently. Here we suggest that not only is there a direct mechanistic link between them, but our understanding of each may be enhanced by considering the other.

The J-C effect describes a process by which specialist pests or pathogens (often soil pathogens?) impose negative conspecific density dependence on tree seedlings, suppressing dominance of abundant species and thus allowing for greater coexistence among species that would otherwise be competitively excluded from a community. The J-C effect is hypothesised as one mechanism by which the tropics supports high tree diversity due to greater strength of biotic interactions at lower latitudes. While J-C effects have been widely reported, empirical evidence for a latitudinal gradient in the strength of conspecific density dependence is mixed. In addition, it remains unclear how higher latitude forests can be dominated by one or a few tree species without succumbing to J-C effects.

There are multiple putative explanations for masting, including non-adaptive explanations such as cycling with resource availability, and various economies of scale, including pollination efficiency and seed dispersal efficiency, whereby individuals benefit through synchronising production of flowers (for pollination) or seeds (dispersal) with nearby conspecifics. However, seed masting was first, and is still most often, associated with escape from seed predators through predation satiation (also an economy of scale). During mast years, when seeds are overabundant, predators consume a lower proportion of seeds (due to satiation), while low seed availability in non-mast years prevents seed predators from maintaining high abundance.

Selection for masting via predator satiation and the J-C effect require various assumptions, and both are predicated on the assumption that seed predators are specialist and that tree recruitment is limited by seed predation. The selective landscape that favours masting might therefore support J-C effects and vice versa although each require additional assumptions (e.g. predators cannot maintain high population densities between masting events and the geographical synchrony in masting matches or exceeds the home range of seed predators else local predator populations would be supplemented by immigration of individuals, while in the J-C effect the spatial scale of seed dispersal must relate to the home range of seed predators). However, we suggest that while intricately linked, masting and J-C effects present two opposing ecological dynamics.

Negative conspecific density dependent J-C dynamics select for wider dispersal kernels, allowing species to escape predation in space. Seedlings establishing far from the mother tree are more likely to find predator/pathogen free space. There is no temporal component to J-C effects. Masting (through predator satiation or reduction in pathogen prevalence) provides a mechanism by which seedlings escape predation in time via manipulating the resource base required to sustain high predator abundance. Masting does not have an explicit spatial component, although there is an implicit assumption that seed predators are aggregated in space. It is perhaps because these processes have been described as operating across distinct axes—space vs time—that they have been considered independently. We believe that greater insight may be gained by considering them jointly.

1. Species coexistence (community diversity): J-C effects facilitate coexistence by suppressing competitive dominance through negative density dependence. However, even in the species-rich tropics there is now increasing evidence that tree communities tend to be dominated by a few hyper-dominant species and most species are rare, indicating that some species are able to overcome the density dependence imposed by J-C effects. We suggest seed masting might allow release from J-C imposed density dependence by reducing pest and pathogen abundance/prevalence. If this were the case, we would expect masting to be more common in ecologically dominant (abundant) species than in rare species.

2. Why is masting not more common? Why some trees mast and others do not remains a puzzle. We suggest that species exposed to strong J-C effects experience strong selection for a wider dispersal kernel, allowing them to escape pests and pathogens in space, and disconnecting the implicit assumption in the predator satiation masting model that seeds (and predators) are aggregated in space. If this were the case, we would expect seed masting to be more prevalent in ecologically dominant trees with shorter dispersal kernels, and rarer in trees with wider dispersal kernels.

3. Latitudinal gradients in species richness: J-C effects have been suggested as contributing to the latitudinal gradient in species richness, with biotic interactions, including seed predation stronger in the tropics. While some empirical evidence supports a latitudinal gradient in biotic interaction strengths (including seed predation), it is not obvious what drives this gradient and how higher latitude forests escape J-C effects. If masting provides a mechanism to escape J-C mediated density dependence, then we might predict that masting would be more frequent in temperate and boreal forests, where community diversity is lower and intraspecific abundance is often very high. We further speculate that selection for species with strongly seasonally structured phenologies in higher latitude forests, might have resulted in tree physiologies that were pre-adapted for the the evolutionary transition to masting, and thus we would predict that not only would masting be relatively more common in high latitude forests but we should also observe a greater frequency of evolutionary independent origins to more periodic reproduction events.

4. Evolutionary trade-off dispersal vs masting (masting has a fitness cost of delayed reproduction – what about dispersal costs?) – is there a fitness landscape across which we can locate species (3d plot).

Extensions:
1. Phylogenetic J-C expanded to consider phylogenetic masting i.e. closely related co-occurring trees should be more likely to mast synchronously assuming host breadth of seed predators is phylogenetically structured (which it is).
2. Pathogens vs predators?
3. Plant-soil feedbacks [I can’t remember where we were going with this …]
4. Spatial/temporal scale …

Future Directions/Recommendations:
How do we combine and test models jointly?


\section{Outline plus homeless notes}

Janzen-Connell (conspecific negative density dependence CNDD) -- spatial density dependence that leads to coexistence; and masting (insert definition)

\begin{enumerate}
\item Outline the dichotomy between these fields (parallel literatures): 
\begin{enumerate}
\item Similar processes: both focused on density dependence (in trees)
\item Both focused on ways to escape pathogens (via space, JC ... and via time masting).
\item Different dimensions: JC focused on space, masting focused on times.
\item Different outcomes assumed
\begin{enumerate}
\item JC is focused on negative DD and masting is focused on positive DD (but the masting is probably cycling over time between positive and negative). 
\end{enumerate}
\item Transition: JC studies have no time, only space, and EOS Masting have looked over time. So both fields of study have ignored one important axes (either of space or time), which means it is difficult to understand the entire landscape
\end{enumerate}
\item What would happen if we brought them together? 
\begin{enumerate}
\item Insert 3D time-space-pressure figure and predictions. 
\item Highlight the Spatial and temporal scale of pathogens/consumers. (Add to host specificity... people need to include this in empirical tests.) When masting works depends on this AND the host specificity. Predict spatial scale of synchrony and depending on the generation time/mobility of consumers. Do JC and/or masting make implicit assumptions of mobility and generation time of consumers that each make? 
\end{enumerate}
\item Things that might get resolved
\begin{enumerate}
\item Hey, you guys explaining masting, could help explain community diversity. And you guys explaining community diversity could explain masting. 
\item Masting people -- are REALLY interested in why trees mast (and how it matters to consumers) and maybe this could all connect to rarity/commonness and coexistence. 
\item Plant soil feedbacks ... somehow
\item Latitudinal gradients? 
\begin{enumerate}
\item JC helps explain latitudinal gradient, but evidence is pretty weak; if masting is more prevalent in northern latitudes, then you might resolve this lack of evidence by ...
\item if masting species are more prevalent outside the tropics then they escape CNDD and this explains latitudinal gradient in diversity (but -- you may think, `hey there's no evidence of less CNDD in temperate' and we say -- people have not looked at space and time properly)
\end{enumerate}
\end{enumerate}
\item Outstanding questions ...
\begin{enumerate}
\item What is supposed to happen with JC over time? (JC should consider temporal aspect more)
\item Is masting as a way to escape JC? JC means whatever is common gets attacked, so if you mast you can avoid this and stay common. Most tropical forest communities are dominated by a couple species, so we predict they mast? 
\item What are the evolutionary implications of JC: Masting is a strategy to escape pathogens, JC you're stuck with pathogens and escape in space -- the strategy then should be to evolve dispersal (back to animal dispersal)... 
\end{enumerate}
\end{enumerate}

\vspace{5ex}
{\bf Notes from 8 May 2024 meeting at KWEER without a home}

Stuff related to what each field needs to do and/or what we want people to do after reading this paper
\begin{enumerate}
\item pull masting to community perspective ... needs phylogeny and pulls towards pest/pathogen host specificity 
\item  Both JC and masting should care more about host specificity of pathogens (with phylogeny of hosts)
\item JC people should think if masting explains anything. 
\item Stop measuring saplings only; look at germinants. 
\item So JC should consider temporal aspect more (from above) 
\item Develop modeling tools that have space and time.
\end{enumerate}


Semi-crazy ideas ... 
\begin{enumerate}
\item If there is phylogenetic signal in pest/pathogen host range then it would select for masting synchrony in those species; so phylogenetic signal currently observed in masting would be RECENTLY evolved (convergent evolution, not common ancestor) ... assumes species overlap in geographic ranges and co-occurrence in a community. 
\item Related to why masting might be more prevalent in temperate zone (is it?) 
\begin{enumerate}
\item  It's easier to develop cues for masting in seasonal habitats? Evolutionary inertia of seasonal cues makes it easier to evolve masting cues? 
\item Seasonality slows down build of soil pathogens after masting? Is it easier to burn out pathogens in the tropics when the pathogens may be less likely to have dormancy stages. So could seasonal (within-year) seed production in the tropics be masting (burns out pathogens) effectively? And thus people don't fully recognize masting in tropics? MERGE this with, is it easier to evolve masting if you're seasonal already.
\end{enumerate}
\end{enumerate}

Masting (an economy of scale) -- predicts positive density dependence over time  
\begin{enumerate}
\item If you are doing it asynchronously then you need host specificty.
\item If you do it synchronously you can have generalist consumers.
\item In between you can maybe less synchronous if hosts are phylogenetically distant. See Parker et al. 2015: `Phylogenetic structure and host abundance drive disease pressure in communities' which cites Janzen-Connell. 
\end{enumerate}

... both verbal models. They only focus on trees, because you need a longer generation time (seed to reproductive maturity) to be longer than the predators can cycle. (They JC people think it works beyond trees -- e.g., grasslands -- and work on trees only because it's cool to explain all the trees.)

% Leaving this out probably (conversation got nowhere): One EOS for masting is pollination efficiency (and does not intersect with JC). In the tropics everything is animal pollinated maybe because (1) rarity, (2) trees have leaves all the time so no wind work and (3). See Annual Review: % https://www.annualreviews.org/content/journals/10.1146/annurev.es.13.110182.002433

% JC says many species in tropics because so many specialized consumers (pathogens) ... and CNDD is stronger in tropics, but now we say little CNDD in temperate zone. (Not sure what I meant here)

\section{Figures}

\begin{enumerate}
\item Look at citations from each foundational paper in field (Janzen and Connell's separate papers, then Janzen's masting paper) and see how much they cite across the two literatures. 
\item 3D time-space-pressure: axes are: time since masting, space from mother tree and predator (pathogen) pressure. JC works across space to say that pressure is HIGH close to mother tree, masting says pressure is LOW in mast years ... but what about in between? Could also use this figure to highlight that JC focuses only on space and masting focuses only on time. 
\item Two separate figures: one for JC people which is seeds per m2 versus survival (spatial) and for the masting it's seeds per m2 versus survival and temporal 
\end{enumerate}

\section{Quick reference notes}

\begin{enumerate}
\item `Feedback with soil biota contributes to plant rarity and invasiveness in communities' by Klironomos. 
\item Hulsman 2024:  \href{https://www.nature.com/articles/s41586-024-07118-4}{Latitudinal patterns in stabilizing density dependence of forest communities}
\item Zhang phylo paper 2022:  \href{https://besjournals.onlinelibrary.wiley.com/doi/full/10.1111/1365-2745.13879}{Phylogenetic dependence of plant–soil feedback promotes rare species in a subtropical forest}
\item Parker 2015: Phylogenetic structure and host abundance drive disease pressure in communities
\item Papers that seem to mention both masting and JC:
\begin{enumerate}
\item Nathan 2004: \href{https://besjournals.onlinelibrary.wiley.com/doi/full/10.1111/j.0022-0477.2004.00914.x}{A simple mechanistic model of seed dispersal, predation and plant establishment: Janzen-Connell and beyond} -- modeling paper
\item Martini 2021 \href{https://besjournals.onlinelibrary.wiley.com/doi/full/10.1111/1365-2745.13833}{Variation in biotic interactions mediates the effects of masting and rainfall fluctuations on seedling demography in a subtropical rainforest} -- explicitly looks for temporal masting effects of JC
\item Bogdziewicz 2018: \href{https://link.springer.com/article/10.1007/s00442-018-4069-7}{Effectiveness of predator satiation in masting oaks is negatively affected by conspecific density} says ``The JC effect and PSH have been so far studied separately (but see Xiao et al. 2016)." ... Xiao Z, Mi X, Holyoak M, Xie W, Cao K, Yang X, Huang X, Krebs CJ (2016) Seed–predator satiation and Janzen-Connell effects vary with spatial scales for seed-feeding insects. Ann Bot 119:109–116
\end{enumerate}
\end{enumerate}

% Storage effect: good and bad environments... 
% Check Usinowicz masting paper...

\end{document}

People who work on this ...
- Ingrid Parker
- Ian Pearse
- Liza Comita

